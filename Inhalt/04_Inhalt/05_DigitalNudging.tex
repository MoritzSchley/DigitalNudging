\chapter{Digital Nudging}
\section{Definition von Digital Nudging}
Jeden Tag trifft ein Mensch durchschnittlich 10.000 Entscheidungen, manche größer, mache kleiner. In der heutigen Zeit, in der eine Welt ohne Technology und Digitalisierung nicht mehr vorstellbar wäre, werden viele dieser Entscheidungen online und somit digital getroffen. Weshalb es keinesfalls verwunderlich ist, dass nudging auch im digitalen Raum verwendet wird. Das sogenannte Digital nudging. Digitales Nudging ist die Verwendung von Elementen der Benutzeroberfläche, um das Verhalten der Menschen in digitalen Entscheidungsumgebungen zu steuern \parencite[]{Weinmann.2016}. Hier liegt der Focus auf dem Design der Benutzeroberfläche. Typische Nudges können hier Mitteilungen sein, die einen an Produkte im Warenkorb erinnern, die noch gekauft werden sollen. Oder ein Fehler, der schon vorhersehbar ist und deshalb verhindert werden soll. Ein Beispiel, dass jeder Erwachsene kennt, ist hier, dass der Geldautomat nur dann das Geld ausgibt, wenn die Karte von Besitzer schon entnommen wurde.
Digital Nudges sollten auch im digitalen Umfeld als eine Hilfestellung dienen, damit Nutzer basieren darauf die für sie Beste Entscheidung treffen können. 
Möglichkeiten, um Nudges hervorzuheben sind grafische Designanpassungen. Dies kann durch Text und Farbe erreicht werden. Anordnungen von Inhalten, sowie auch durch kleine Animation, damit die Aufmerksamkeit auf die Digital Nudges gelenkt wird.

\section{Ethische Grundlagen}
Die Betrachtung der ethischen Grundlagen bildet das Fundament für die Diskussion über Nutzen, Chancen und Risiken von Digital Nudging, vor allem im ökonomischen Umfeld. Zudem ist sie notwendig, um einen Nudge von einer Manipulation bzw. Überredung des Users abzugrenzen \parencite[S. 3]{Jesse.2021}. Das ethische Fundament von Nudging ist nicht exklusiv für den digitalen Rahmen anzuwenden, jedoch ist die Einhaltung hier wichtiger.

Laut Kahnemann spricht die Nutzung von mobilen Anwendungen bei einem Großteil der Bevölkerung das intuitive Denksystem, auch System 1 genannt, an. Dies steht im Gegensatz zum rationalen und überdenkenden System 2, welches Viele bei Kaufentscheidungen im Laden nutzen. Durch die vorherrschende Nutzung von System 1 handeln Personen im digitalen Umfeld impulsiver und mit weniger Überdenkzeit, \parencite[Absatz 1b]{Kahneman.2012} was die Wirksamkeit von Nudges erheblich steigert. Zudem vereinfachen die Einsatzmöglichkeiten von Algorithmen und die schnelle Anpassungsfähigkeit von User Interfaces die Individualisierung von Nudges, sowie das Testen deren Wirksamkeit \parencite[S. 88]{Reisch.2020}. So teilt Instagram beispielsweise ihre User in zwei Gruppen auf, um Interface-Änderungen zu testen. Bei einer Hälfte wird das Update ausgespiel, die andere Hälfte bleibt auf der bereits eingesetzten Software-Version. Damit ist es Instagram möglich, einen direkten Vergleich zwischen beiden Oberflächen zu haben, und Rückschlüsse mit minimaler Verzerrung zu ziehen.
\subsection{Autonomität}
Die wesentliche ethische Grundlage ist die Gewährleistung von Autonomität des Users bei der Konfrontation mit einem Nudge. Das bedeutet, dass die uneingeschränkte Wahlfreiheit des Nutzenden zwischen allen Optionen gewährleistet werden muss. \parencite[S. 88]{Reisch.2020} Jedoch ist die freie Wahl nicht nur auf die Verfügbarkeit aller Optionen zu beziehen, sondern die Entscheidung des Nutzenden sollte bedacht, und ohne erheblichen Einfluss von Dritten getroffen werden. \parencite[S. 6]{Lembcke.2019}

Ein solche Ausgangssituation ohne externe Einflüsse ist oft nicht komplett vermeidbar, da diese nicht von der Plattform beeinflusst werden können. Jedoch sollten die Einflüsse innerhalb der Anwendung auf ein Minimum reduziert werden. \parencite[S. 88]{Reisch.2020}Wenn man im Browser nach Hilfsmitteln gegen Kopfschmerzen sucht, bekommt man häufig Websites von bekannten Marken vorgeschlagen. Diese informieren meist zuerst über die Ursachen von Kopfschmerzen, und geben danach Tipps zur Linderung der Symptome, welche die Produkte der Marke bewerben. Aus eingener Nachforschung lässt sich erkennen, dass Nutzende dadurch zum Kauf bewegt werden, ohne Alternativen abzuwägen. Dies ist aus Sicht der Marke sinnvoll, schränkt allerdings in gewisser Weise die Autonomität des Users ein. Zwar gibt es keine Auswahl an verschiedenen Anbietern, jedoch steht es dem Besuchenden der Website frei, ob er das Produkt tatsächlich kaufen möchte.

Wenn bei der Betrachtung eines Nudges die Autonomität nicht eindeutig festzustellen ist, bzw. diese von der Perspektive abhängig ist, kann die ethische Fundierung anhand der anderen Grundlagen bestimmt werden.
\subsection{Transparenz}
Die Transparenz eines Nudges ist gegeben, wenn die benutzende Person den Einfluss durch den Nudge, sowie die Ziele des Nudges, erkennt. Zudem sollten alle Optionen sichtbar und als Solche zu erkennen, sowie die Bedeutung dieser Optionen klar verständlich sein. \parencite[S. 89]{Reisch.2020} Dadurch steht die Transparenz im direkten Zusammenhang mit der Autonomität des Nudges. Wenn eine Auswahl an Optionen besteht, diese allerdings nicht klar sichtbar sind, ist die Transparenz nicht sichergestellt, und somit durch mangelnde Auswahl auch die Autonomität eingeschränkt.
\subsection{Zielrechtfertigung}
Nudges können, je nach Ort und Art des Einsatzes, verschiedene Ziele verfolgen. Um ethisch korrekt zu sein, müssen soziale oder nutzerbezogene Ziele verfolgt werden. Eigennützige Ziele aus Sicht des Providers sind selten ethisch zu rechtfertigen, außer es ist nachweisbar, dass die Ziele mit denen eines Großteils der Nutzenden übereinstimmen. Im Gegensatz dazu, können soziale Ziele sogar dann ethisch korrekt sein, wenn Nutzende ihnen nicht zustimmen, solange kein Beeinflusster den Einfluss als einschränkend oder untragbar ansieht. \parencite[S. 89]{Reisch.2020}\parencite[S. 7]{Lembcke.2019} In diese Kategorie fallen vor allem Self-Nudges, die den Nutzenden zu einem besseren Verhalten bewegen, wie zum Beispiel Aufforderungen für Spenden. Auch die von Google Maps vorgeschlagene Option, den kraftstoffsparendsten und damit umweltfreundlichsten Weg für die Autofahrt anzuzeigen, kann als diese Art von Nudge kategorisiert werden.