\chapter{Nudging}
\section{Definition}
Übersetzt bedeutet Nudging „sanfter Stups“. Ein Sanfter Stups in die richtige Richtung, um eine Entscheidung in eine vorhersehbare Weise beeinflussen zu können. 
Der Begriff kommt ursprünglich aus der Verhaltensökonomie und wurde vor allem durch Richard Thaler und Cass Sunstein maßgeblich geprägt. Richard H. Thaler wurde 1945 geboren und war ein US-amerikanischer Wirtschaftswissenschaftler, Autor sowie Professor für Verhaltens- und Wirtschaftswissenschaften an der University of Chicago Booth School of Business. Cass R. Sunstein wurde neun Jahre später, 1954, geboren. Er war ein US-amerikanischer Wirtschaftsrechtswissenschaftler mit zusätzlichem Interesse an Verhaltensökonomie, lehrte fast drei Jahrzehnte lang an der University of Chicago Law School und war in der Obama-Regierung im White House Office of Information and Regulatory tätig.
Das Nudging-Konzept der Beiden nutzt Elemente der Verhaltensökonomie und psychologische Erkenntnisse, um das Verhalten von Marktakteuren zu beeinflussen. \parencite[]{Kenning.2016}
Das Ziel von Nudging besteht darin, eine höhere Effektivität und eine höhere Effizienz bei politischen Strategien und Maßnahmen zum Wohle der Gesellschaft, aber auch des einzelnen Verbrauchers zu erreichen.
Hier ist wichtig zu realisieren das es elementare Unterschiede zwischen Manipulation und Nudging gibt. Wenn ein Navigationsgerät die nächste Tankstelle anzeigt, ist das als Nudge einzuordnen. Wenn allerding nur Tankstellen von Sponsoren oder Partnerunternehmen angezeigt werden, obwohl diese weiter entfernt sind, kann das als Manipulation angesehen werden. \parencite{Hansen.2017} Die genauen Grundlagen, welche Nudging ausmachen, werden in einem späteren Teil dieser Arbeit im Kapitel ethische Grundlagen genauer beleuchtet.