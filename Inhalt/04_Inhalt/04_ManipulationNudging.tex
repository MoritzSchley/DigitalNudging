\chapter{Ist Nudging Manipulation?}

\section{Definition von Manipulation}
Unter Manipulation wird im weitesten Sinne der Versuch eines Lebewesens verstanden, das Verhalten oder die Wahrnehmung eines anderen Lebewesens zu beeinflussen oder zu ändern. \parencite[]{Horn.2019} Wenn über Manipulation geschrieben wird, ist dies meist positiv oder negativ konnotiert. Dies dient der Differenzierung von missbräuchlicher gegenüber weniger schwerwiegender Manipulation. Unter Letzteres fällt beispielsweise die Überredung. Thaler und Sunstein plädierten durchwegs für eine ethische Nutzung von Nudging. Ob dies eingehalten wird kommt auf den Choice Architect und auf die Absicht hinter dem Nudge an. 

\section{Differenzierung von Nudging und Manipulation}
Beim Nudging handelt es sich laut Thaler und Sunstein nicht um Manipulation. Dies machen sie durch ihre starke Distanzierung ihrer Theorie von manipulativer Nutzung. Dies fängt schon mit ihrer Leitphilosophie Libertärer Paternalismus an. Hier setzen sie wohl weislich „Libertärer“ also „Autonom“ davor. Doch immer wieder kommt die Debatte auf, dass es sich bei Nudging um Manipulation handelt. Wenn zum Beispiel in Kalifornien die Regierung durchsetzt, dass Nachbarn gegenseitig ihren Stromverbrauch gezeigt wird hat dies das Ziel, den Verbrauch zu senken. Das ist noch nicht als klare Manipulation erkennbar, sondern kann als ein Nudge in Richtung umweltbewusstes Verhalten gesehen werden. Argumente für die Klassifizierung als Nudge sind die Freiwilligkeit, da Haushalte mehr Strom verbrauchen können, wenn gewollt. Andererseits kann man es als soziale Manipulation ansehen, da ein hoher Verbrauch von den Nachbarn, und daher vom gesellschaftlichen Umfeld, verurteilt wird. \parencite{BerufsverbandDeutscherPsychologinnenundPsychologen.2015}. 
Ein anderes Beispiel, welches viele Männer in Urinalen schon miterlebt haben, ist das der Plastikfliege im Urinal. Die meisten Männer versuchen hier die Fliege zu treffen, weshalb 80\% weniger Urin daneben geht auf öffentlichen Toiletten. Dies ist generell nicht als Manipulation angesehen zu werden, müsste der Einfluss unterbewusst stattfinden. Zudem ist der Einsatz solcher Hilfen gerechtfertigt, da eine saubere Toilette im Interesse der Öffentlichkeit ist \parencite{Brautzsch.2020}.
Um über die zukünftige Nutzung von Nudging zu entscheiden, müssen klare Unterschiede zu eigennütziger Motivation definiert und eingehalten werden. Aktuell sind die formalen Grenzen zwischen diesen allerdings noch sehr verschwommen.
Bei der Verwendung von Nudging im kommerziellen Umfeld ist es wichtig, dass Choice Architects nicht mit der Intention der Manipulation handeln und diese als Nudging tarnen. Stattdessen sollte das Nudging stets mit der Absicht verwendet werden, das Wohl der Kunden und ihre Erfahrung positiv zu beeinflussen. Wird Nudging mit eigennützigen oder primär profitorientierten Intentionen verwendet, fällt es ebenfalls in den Bereich der Manipulation.
Ob Nudging unter Manipulation läuft, wird wahrscheinlich für das Bestehen des Nudgings eine umstrittene Frage sein, auf die viele mit Ja und andere mit Nein antworten. 
Das Wichtigste ist, dass wenn Nudging verwendet wird, der Choice Architect nicht mit der Intention der Manipulation, sondern mit der Intention der Entscheidung Hilfe zum Wohl des anderen handelt. Denn Nudging kann als Manipulation verwendet werden, wenn mit Bösen Absichten gehandelt wird.
