\chapter{Fazit}
Wie in den vorausgegangenen Kapiteln erarbeitet, ist Digital Nudging und seine Wirkung stark davon abhängig, wie und wer es einsetzt. Als Hilfestellung im e-commerce bietet Digital Nudging Vorteile für Kunden, und damit einhergehend einen höheren Profit für das Unternehmen. Nudges können als Self-Nudges Angewohnheiten und Verhalten von Nutzenden verbessern, bei Spendenportalen die Quote an Spendenden erhöhen und oder durch Standardeinstellungen bei z.B. Google Maps Personen dazu bewegen, umweltfreundlicher zu fahren.

Wenn die ethischen Grundsätze jedoch missachtet werden, tritt schnell Unzufriedenheit bei den Kunden auf, und ein negatives Bild gegenüber dem Unternehmen entsteht. Es ist zwar möglich, durch Cookie-Banner oder auch Voreinstellungen zum Abonnieren von Newslettern bei der Bezahlung kurzfristig mehr Gewinn zu generieren, jedoch entwickeln Kunden unterbewusst eine Abneigung gegen das Unternehmen, welche dem langfristigen Unternehmenserfolg schaden können.

Aus Interesse an einer freien Entscheidungsbasis ist eine Überarbeitung der Gesetze, sowohl national als auch international, notwendig. Diese sollten den positiven Einsatz von Nudges nicht einschränken, aber die Nutzung für rein kommerzielle, beziehungsweise eigennützige Ziele unattraktiver machen.