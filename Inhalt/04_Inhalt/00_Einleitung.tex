\chapter{Einleitung}
Welpen, Obst und Gemüse auf Augenhöhe, Fußabdrücke in öffentlichen Gebäuden, kleinere Teller in Kantinen, aufgeklebte Plastik Fliegen in Urinalen, Kalorienanteil und Nutri-Score auf Lebensmitteln, Opt-Out Orangespende Regeln, sowie viele andere Dinge in unserem Alltag, sind Beispiele für Nudging. Es ist fast unmöglich für einen Menschen nicht mit Nudging in Kontakt zu kommen und noch schwerer, nicht von ihnen beeinflusst zu werden. In der folgenden Seminararbeit wird unter anderem fundiert dargelegt, was man unter dem Begriff Nudging allgemein versteht, auf welchen wissenschaftlichen Erkenntnissen Nudging beruht, welche wissenschaftliche Disziplin wesentliche Grundlagen geleistet hat und leistet. Dies alles wird mit praktischen Beispielen verdeutlicht Darüber hinaus wird im zweiten Teil erläutert, was genau man unter Digital Nudging versteht, worin der Nutzen für Unternehmen und Kunden besteht und anhand von Beispielen erläutert wie die Anwendung des Digital Nudging in der Praxis möglich ist. Der Abschluss der Seminararbeit besteht aus einer kritischen Diskussion der Chancen und Risiken des Digital Nudgings.

Am nachfolgenden Text haben Jana Walcher, Moritz Schley, Arkin Cip und Paulina Kohlhepp in Zusammenarbeit gearbeitet.