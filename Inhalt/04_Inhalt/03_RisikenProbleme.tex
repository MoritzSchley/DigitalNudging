\chapter{Ethische und rechtliche Probleme bei Digital Nudging}
Dieses Kapitel diskutiert Risiken und Probleme, die bei der Verwendung von Digital Nudging auftreten können.
\section{Helles und dunkles Nudging}
Bei der Nutzung von Nudges im digitalen Bereich lässt sich grundsätzlich zwischen hellem und dunklem Nudging unterschieden. Dabei beschreibt helles Nudging einen nach den ethischen Grundlagen "guten" Zweck, während dunkles (dark) Nudging für Zwecke steht, die gegen die ethischen Grundlagen verstoßen.

Wichtig ist dabei, dass Nudging als Instrument in der Diskussion erst einmal neutral ist. Erst durch die Definition eines Anwendungsfalls kann darüber diskutiert werden, ob es sich um einen hellen oder dunklen Nudge handelt.

Dennoch lässt sich in der Wirtschaft abzeichnen, dass Nudging zunehmend für unethische Zwecke verwendet wird. Dafür haben sich die Begriffe ``Dark Nudging'', ``Sludges'' und ``Dark Patterns'' etabliert. \footfullcite[S. 73-74]{Narayanan.2020} Der Begriff Dark Patterns wurde dabei 2010 erstmals von  Brignull definiert und beschreibt Tricks in Webseiten und Apps, die Nutzer:Innen zu Handlungen bewegen, die von ihrer eigentlich gewünschten Handlung abweisen. \footfullcite{Brignull.2010}

Wissenschaftler haben herausgefunden, dass Dark Patterns in über 1200 Shopping Seiten \footfullcite[S. 2]{Mathur.2019} und in mehr als 95\% von beliebten Android Apps \footfullcite[S. 5]{DiGeronimo.2020} implementiert sind. Daran lässt sich erkennen, dass Dark Nudging sehr in den Alltag von Nutzerinnen und Nutzern eingreift und kein Randphänomen ist.

Thaler, der wie in Kapitel (HIER EINSETZEN) die Theorie hinter Nudging eingeführt hat, distanziert sich mittlerweile öffentlich von der Dark Patterns Nutzung seiner Idee. Er nennt unethische Nudges ``Sludges''. Seiner Ansicht nach sollten Nudges nur genutzt werden, um die Umgebung, in der Menschen Enrscheidungen treffen angenehmer zu gestalten. Dies ermögliche selbstbewusstere Entscheidungen. Dark Pattern beschreibt er als "nudging for evil". \footfullcite{Thaler.2018}

\section{Rechtliche Situation}
Die Präsenz des Themas im alltäglichen Gebrauch lässt auch die Gesetzgebung in Deutschland und der \ac{EU} auf das Thema aufmerksam werden. Auch wenn es bisher keine einheitliche Regelung bezüglich Digital Nudging und Dark Patterns gibt, so sind in Einzelfällen (wie bspw. in HIER KAPITEL EINSETZEN) bereits rechtliche Grundlagen zur Nutzung von Nudges gesetzt worden.

Der Gewinnspielbetreiber "Planet49" hatte bei seinem Cookie Banner standardmäßig alle Cookies mit Ankreuzkästchen aktiviert. Zum Ablehnen der Datensammlung war es notwendig, die Häkchen nach und nach anzuklicken. Der Bundesverband der deutschen Verbraucherzentralen und Verbraucherverbände hat gegen diese Implementierung geklagt. In einem Urteil des \ac{EuGH}, welches später durch den deutschen \ac{BGH} bestätigt wurde, wird erläutert, dass eine wirksame Einwilligung nicht durch vorausgewählte Einstellungen erfolgen könne. Außerdem seien zu einer vollständigen Einwilligung weitere Informationen notwendig, wie z.B. die Funktionsdauer der Cookies oder mit welchen Drittanbietern sie geteilt werden. \footfullcite{Hartung.2020}

Aktuell befasst sich die \ac{EU} im Rahmen des \ac{DSA} mit Dark Patterns. So plant die \ac{EU}, Dark Patterns weitreichend zu verbieten. ``Anbietern von Online-Plattformen sollte es [...] untersagt sein, die Nutzer in die Irre zu führen oder zu etwas zu verleiten und die Autonomie, die Entscheidungsfreiheit oder die Auswahlmöglichkeiten der Nutzer durch den Aufbau, die Gestaltung oder die Funktionen einer Online-Schnittstelle oder eines Teils davon zu verzerren oder zu beeinträchtigen.'' \footfullcite[S. 18]{EuropaischeUnion.2022} Damit gibt es erstmalig eine Rechtsgrundlage zur allgemeinen Benutzung von Nudes im digitalen Raum. Dennoch gibt es Bedenken, dass die Forderungen des \ac{DSA} nicht weitreichend genug seinen, etwa weil sich der die Regelungen nur auf ``Online-Plattformen'' beziehen, worunter nicht alle Internetseiten fallen. \footfullcite{King.2022} Es bleibt abzuwarten, wie effektiv die Regelungen des \ac{DSA} sind und ob Plattformbetreiber ihre Inhalte nach Inkrafttreten der Verordnung in den Mitgliedsstaaten anpassen.