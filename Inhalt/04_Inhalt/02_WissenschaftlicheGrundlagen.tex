\chapter{Wissenschaftliche Erkenntnisse zu Nudging}
Mit dem 2008 veröffentlichtem Buch ``Nudge: Improving Decisions About Health, Wealth, and Hapiness'', geschrieben von Richard H Thaler und Cass R Sunstein, wurde der Grundstein für die Nudge Theorie gelegt. Thaler und Sunstein verweisen bei ihrer Definition der Nudge Theorie sich im Großen und Ganzen auf die Prospekt Theorie, definiert von Daniel Kahneman und Amos Tversky. Aus diesem Grund wird diese, als auch die vorhergehende Expected Utility Theorie, im Weiteren näher beschrieben und die wissenschaftlichen Erkenntnisse, welche hervorgingen definiert. 

\section{Expected Utility Theorie}
Die Expected Utility Theorie (Auch EUT genannt) wurde als deskriptives Model für das ökonomische Verhalten von Menschen verwendet. Es beschreibt, wie sich Menschen verhalten würden, wenn sie bei wirtschaftlichen Entscheidungen rational vorgehen würden. \parencite[S. 279]{Friedman.1948}  Grundsätzlich geht es bei den Entscheidungen, welche in der Expected Utility Theorie untersucht werden, um eine Wahl zwischen zwei Optionen, die beide Monetär und mit einer bestimmten Wahrscheinlichkeit versehen sind. Als Ergebnis der Untersuchungen wurden drei Grundannahmen festgelegt.  

Die erste Annahme definiert, wie hoch der erwartete Nutzen einer Option ist. Allgemein ist laut der EUT der Nutzen einer Option gleichgestellt mit dem erwarteten Nutzen des Ergebnisses der Option. Die zweite Annahme ist die Anlagen-integration. Diese besagt, dass eine Option akzeptabel ist, falls der resultierende Nutzen, der durch die Wahl der Option erlangt wird, größer ist als der Vermögenswert ohne Wahl der Option. Die dritte Annahme stellt die Risiko-vermeidung dar. Sie besagt die das Menschen risikoscheu sind, und eine sichere Option immer einer riskanteren Option vorziehen. Diese Annahme führte dazu, bei weiteren Untersuchungen davon ausgegangen wurde, dass dadurch auch die Nutzen-funktion konkav ist. \parencite[S. 127]{Pratt.1964}

Kahneman und Tversky haben bei ihren Untersuchungen zur EUT einige Erkenntnisse gesammelt und als Kritik an jene Theorie vermittelt. Die Untersuchungen wurden in Form von quantitativen Umfragen durchgeführt, bei welchen die Probanden über monetäre Entscheidungen befragt wurden. Durch das Anpassen der Versuchsumgebung konnten die verschiedenen Wissenschaftlichen Erkenntnisse abgeleitet werden, welche die Grundlage für die Prospekt Theorie darstellten. Der erste Effekt, welcher definiert wurde, ist der Spiegelungseffekt. Dieser besagt, dass bei Entscheidungen zwischen zwei Optionen, welche sich beide Auswirkungen im positiven Wertebereich haben, risikoscheu gehandelt wird. Jedoch bei zwei Optionen im negativen Wertebereich, risikofreudig entschieden wird. \parencite[S. 154]{Markowitz.1952}

Beispielsweise wird bei einer Entscheidung zwischen einem sicheren Gewinn von 3000 oder einem Gewinn von 4000 mit einer Wahrscheinlichkeit von 80\%, die Mehrzahl die sicheren 3000 wählen.\parencite[S. 266]{Kahneman.2013} Allerdings würden bei den Optionen von einem sicheren Verlust von 3000 oder einem Verlust von 4000 mit einer Wahrscheinlichkeit von 80\%, die Mehrheit die möglichen 4000 wählen.  

Weiterführend spielt hier der Gewissheitseffekt eine beitragende Rolle. Der Gewissheitseffekt besagt, dass ein sicheres Ereignis in der Regel überschätzt wird. Das heißt das sichere Gewinne als sehr stark akzeptiert werden, jedoch sichere Verlust stärker abgelehnt werden. Dies trägt dem Spiegelungseffekt bei. 

Ein weiterer Effekt, welcher eine wichtige Erkenntnis darstellt, ist der Isolationseffekt. Dieser sagt aus, dass Menschen Gemeinsamkeiten bei einer Entscheidung außen vorlassen. \parencite[S. 296]{Tversky.1972} Ein Paar an Optionen einer Entscheidung können in zwei verschiedene Teile getrennt werden. Einen gemeinsamen Teil, welcher in beiden Optionen identisch ist, und einen unterschiedlichen Teil. Da die Grenzen dieser beiden Teile nicht eindeutig definiert werden kann, ist es möglich verschiedene Arten an Trennweisen für ein und dieselben Optionen zu haben. Kahneman und Tversky beobachteten, dass bei verschiedenen Aufteilungen, das Ergebnis der Befragung sich verändern kann. 

Als Beispiel wird ein Spiel genommen, bei welchem es eine 75\% gibt bei der ersten Runde in die Möglichkeit zum Spielen in der zweiten Runde zu erlangen. Falls die 75\% zutreffen, hat man die Wahl zwischen einem sicheren Gewinn von 3000 oder einem Gewinn von 4000 mit einer Wahrscheinlichkeit von 80\%. Die meisten Probanden isolieren nun den gemeinsamen Teil der Frage von dem Rest ab. Die erste Wahrscheinlichkeit von 75\% wird ignoriert, da sie für beide weiteren Optionen gleich ist. Die Mehrheit der Befragten entschied sich hier für die Option mit den sicheren 3000. Berechnet man jedoch die erste Runde wieder per Multiplikation in die zwei Optionen hinein, verändern sich die Wahrscheinlichkeiten für die verschiedenen Optionen. Denn schlussendlich ist es eine Entscheidung zwischen einem Gewinn von 4000 mit einer Chance von 20\% oder 3000 mit einer Chance von 25\%. Als dies aber als eigene Entscheidung mit den dazugehörigen Wahrscheinlichkeiten formuliert wurde, fiel die Mehrheit die 4000 als bevorzugte Option.  

Dieses Experiment gab zwei Schlüsse mit sich. Der erste lautet, dass bei einer Entscheidung von zwei Optionen mit gleichem Startkapital, eine Option mit einem fixen Ertrag, attraktiver ist als einem risikoreicherem mit möglicherweise höherem Ertrag. Der zweite sagt grundsätzlich aus, dass Präferenzen von Menschen durch eine unterschiedliche Art der Darstellung beeinflusst werden kann. 

Diese Erkenntnis wurde weitergeführt und es wurde schlussgefolgert, dass das gleiche Prinzip auch für verschiedene Arten der Darstellung der Ergebnisse der Optionen gilt. Bei zwei Fragestellung wurden die Endwerte zweier Optionen in einer positiven Schreibweise und in einer negativen Schreibweise dargestellt. Hierbei ist sind die Endwerte exakt gleich in beiden Fragestellungen. Trotz dessen findet der vorher genannte Spiegelungseffekt statt. Bei der positiven Ausdrucksweise wird die sichere der Beiden Optionen gewählt und bei der negativen Ausdrucksweise die riskantere. \parencite[S. 272]{Kahneman.2013} 

Dies führte zu der Erkenntnis, dass oft nicht der finale Stand nach der Auswahl der Optionen ausschlaggebend für die Wahl ist, sondern die Veränderung der Werte einen höheren Einfluss hat. \parencite[S. 156]{Markowitz.1952}

\section{Prospect Theory}
Die Prospect Theorie ist die Weiterführung der EUT, bei welcher versucht wurde, die vorher aufgeführten Effekte, wie den Spiegelungseffekt oder den Isolationseffekt. Der Fokus lag hierbei auf einfachen Entscheidungen zwischen genau zwei Optionen, welche erneut monetär und mit einer Wahrscheinlichkeit versehen sind. 
Da diese laut Kahneman und Tversky jedoch auf kompliziertere Entscheidungen angewendet werden kann, wurde dies im Falle der Nudge Theorie durchgeführt. \parencite[S. 274]{Kahneman.2013} 
Die Prospect Theorie kann in zwei Phasen aufgeteilt werden, die Bearbeitungsphase und die Bewertungsphase. In der Bearbeitungsphase werden verschiedene Möglichkeiten aufgezeigt, Optionen zu verändern oder neu zu sortieren, um die darauffolgende Bewertung zu beeinflussen. Diese Werkzeuge wurden im Weiteren Verlauf als Grundlage für Nudges verwendet, weshalb sie nun in den Fokus gelegt werden. Die erste Operation stellt das Coden dar. Da eine Erkenntnis ist, dass Menschen Ergebnisse vor allem als die relative Veränderung wahrnehmen anstelle des Finalen Wertes, kann dies auch zur Beeinflussung der Entscheidung verwendet werden. Eine Veränderung wird immer relativ zu einem Referenzpunkt betrachtet. Als zweites kann die Kombinierung verwendet werden. Hierbei werden bei Optionen, Gewinne oder Verluste, die gleich sind, zusammengefasst. Beispielsweise wird so aus einer Option, welche ein Gewinn von 200 mit einer Wahrscheinlichkeit von 20\% oder ein anderer Gewinn von 200 mit der Wahrscheinlichkeit von 20\% zusammengefasst zu, einem Gewinn von 200 mit der Wahrscheinlichkeit von 40\%. Dann gibt es noch die Trennung, bei welcher sichere Anteile der Optionen von Risikoreicheren Anteilen getrennt werden. Nun kommen Operationen, welche auf mehrere Optionen zugleich angewendet werden. 
Die Aufhebung greift auf den Isolationseffekt zurück, um Gemeinsamkeiten der Optionen im vornehinein schon aus der Entscheidung zu streichen. 
Die Simplifizierung setzt auf den Einfluss einer Rundung der Wahrscheinlichkeiten oder der Wertveränderungen. Als Beispiel wird ein Gewinn von 101 mit einer Wahrscheinlichkeit von 49\% von den meisten Personen direkt als ein Gewinn von 100 bei 50\% wahrgenommen. Dies kann im Vergleich zu anderen Optionen genutzt werden, um die Attraktivität der Varianten zu verändern. 
Viele Entscheidungen der Probanden, die von der Rational agierenden EUT abweichen, sind auf eben diese Operationen in der Bearbeitungsphase zurückzuführen. \parencite[S. 46]{Tversky.1969}
Kahneman und Tversky definierten, dass Optionen nicht mehr wie in der EUT nur den erwarteten Wert als Nutzens wert erhalten sollten, sondern stattdessen jeweils ein Paar aus Wert und Gewichtung bilden sollten.
Der Wert einer Option wird mithilfe einer Wertefunktion bestimmt, welche aus zwei Attributen besteht. Ein Referenzpunkt von dem die Veränderung ausgeht und der Ausschlag der Veränderung selbst. Dies liegt daran, dass die Wahrnehmungsorgane des Menschen mehr darauf ausgelegt sind, den Ausschlag einer Veränderung zu bestimmen als den absoluten Wert. Das gilt nicht nur für monetäre Veränderungen sondern auch für andere Attribute wie Lautstärke, Helligkeit oder Temperatur. Ebenfalls ist es einfacher zwischen zwei kleinen Veränderungen zu unterscheiden, als zwischen zwei größeren. Beispielsweise kommt der Menschlichen Psyche der Unterschied zwischen einem Gewinn von 100 oder 200 größer vor als zwischen einem Gewinn von 1100 oder 1200. \parencite[S.278]{Kahneman.2013} Kahneman und Tversky beobachten vor allem in der Monetären Dimension, dass sich die Auswirkung von Gewinn und Verlust bei Menschen Unterschiedlich bemerkbar machen. In der EUT wurde schon von Risikovermeidung im positiven Bereich und Risko Bereitschaft im negativen Bereich geschrieben. Nun kam heraus, dass die Wertefunktion im Negativen Bereich Steiler ist als im Positiven. Das liegt daran, das der gleiche Betrag als Verlust einen höheren Einfluss hat als Gewinn.

Der zweite Teil der Option ist der gewichtete Teil. Dieser wird mit einer Gewichtungsfunktion bestimmt. Die Gewichtung gibt an, wie erstrebenswert eine Option ist, und hat demnach nichts mit der Wahrscheinlichkeit der Option zu tun. Dieses Gewicht wurde in die Bewertung der Optionen übernommen, da es einen großen Einfluss auf die Entscheidung der Probanden hat. Hierbei ist zu notieren, dass Menschen oftmals zum Übergewichten oder Überschätzen neigen. Ersteres passiert deutlich bei besonders kleinen Wahrscheinlichkeiten. Bei einer Entscheidung zwischen einem sicheren Gewinn von 5 oder einem Gewinn von 5000 mit einer Wahrscheinlichkeit von 0,1\% haben 72\% der Probanden die zweite Option gewählt. \parencite[S. 154]{Markowitz.1952} Rational betrachtet ist der Erwartungswert beider Optionen bei 5, allerdings wurde die sehr geringe Wahrscheinlichkeit für den höheren Gewinn übergewichtet. 

Wenn aus den Gewinnen nun aber Verluste gemacht werden, dreht sich das Ergebnis um, und die Mehrheit wählt den sicheren Verlust von 5 anstelle der sehr geringen Wahrscheinlichkeit von einem Verlust von 5000.

Eine weitere Erkenntnis ist, dass die Fähigkeit, extreme Wahrscheinlichkeiten einzuschätzen und zu bewerten stark begrenzt ist. Dadurch werden sehr unwahrscheinliche Ereignisse entweder komplett ignoriert oder Übergewichtet. Das gleiche gilt für sehr hohe Wahrscheinlichkeiten. Der Unterschied zwischen einem sicheren und einem sehr wahrscheinlichen Ereignis wird ebenfalls entweder ignoriert oder überschätzt. 

Ebenfalls wird in der Prospect Theorie ein Augenmerk auf den angesprochenen Referenzpunkt gelegt. Denn durch das Verändern des Referenzpunktes der Optionen, kann auch die Attraktivität der Option beeinflusst werden. Ein Beispiel hierzu wäre, dass es einen Verlust von 2000 gab und man nun zwei Optionen hat. Entweder einen sicheren Gewinn von 1000 oder einen Gewinn von 2000 zu 50\%. Laut der Prospect Theorie würde die Mehrheit bei diesen beiden positiven Optionen, den Sicheren von beiden, also die 1000 wählen. Dadurch hätte man am Ende nur noch einen Verlust von -1000. Ändert man jedoch den Referenzpunkt auf den Stand, dass man sich noch nicht mit dem Verlust der 2000 am Anfang abgefunden hat, so werden aus den zwei Optionen entweder ein sicherer Verlust von 1000 oder ein Verlust von 2000 zu 50\%. In diesem Fall würde auf Grund des Spiegelungseffektes die Mehrheit die Wahl auf dem Risikoreicheren, also dem möglichen Verlust von 2000 zu 50\% fallen. \parencite[S. 288]{Kahneman.2013}